\chapter{Introduction and Background}

\section{The Spaghetti Syndrome and its Implications - the need for a Wireless Vital Signs Monitoring System}

Modern surgeries are challenging working environments involving a combination of complex devices, computers and humans working under stressful and time critical conditions. However, to date there are far too many wired devices communication between machines, monitoring patient physiological conditions and still requiring considerable attention by experts for monitoring and decision making.

In an era where surgery is ubiquitous, an ever increasing number of devices are attached to critically ill patients, multiplying the number of wires and tubes connected to the patient which results in a net resembling a plate of spaghetti; hence, the coinage of the term the Spaghetti Syndrome \cite{imhoff2004spaghetti}. This conundrum has been present in the context of critical care, anaesthesia, and operating theatres for almost three decades with the advent of new technological advances in medicine \cite{cesarano1979spaghetti}. This issue clearly illustrates the need for a system which is not physically connected to avoid possible harm and injury to both the surgeons and patients in the operating theatre. 

The presence of wires in the surgery room introduces the possibility of increased surgery duration due to tripping hazards. More severe trips could lead to further surgical complications, due to wire disconnections which results in inaccurate readings of vital signs. Consequently, a possible overdose or underdose of anaesthetic agents could follow, leading to patient injury, and in the worst case scenario, the demise of the patient. Indirectly, wires in the operating theatre increase the rate of mortality during surgeries. 

One possible course of action to rectify this problem is to remove cables connected to patients by replacing the medium of transmission from a wire to electromagnetic waves. In recent years, multiple research projects have been carried out with the aim to embed vital sign sensors with wireless technology to separate patients from cables \cite{rosenthal2003new}.  

Nevertheless, most advances in this area of wireless device development is limited to the context of medical centres and not operating theatres.  

\section{Benefits of a Wireless Implementation}

Having a wireless version of current vital signs monitoring systems will bring additional benefits to both the patient and the surgeons. One of the biggest advantages is the improved safety factor, leading to less risk of injury due to tripping hazards or disconnection issues. Less obstruction from the presence of wires means smaller delays for operations, making surgeries faster and more efficient. This translates to a better overall working environment for doctors and medical staff. 

The ease of use and portability of wireless systems lead to convenience for patient transportation. Current sensors are connected to multiple screens and this introduces the issue of space constraint. An integrated wireless system allows generic sensors with the same interface to be connected to a single terminal and monitor, circumventing the issue of having multiple displays. 

\section{Overview of the Project}

This  paper will  consider the  process and challenges of  developing a fully integrated wireless vital signs monitoring system for vital signs specifically for anaesthetic parameters such as electrocardiography (ECG), photoplethysmography (PPG), electroencephalography (EEG), blood pressure, and blood temperature. The  front end of the project encompasses the collection and processing of raw data from different sensors including the visualisation and presentation of vital signs information,  according  to some specified performance criterion. The back end of the project involves the wireless transmission and reception of the processed information to a fixed output display and possibly, a secure database. 

The aim of this project is to make as many patient monitoring, measuring devices as fully wireless so enabling more freedom of movement for both patients, nurses, and medical staff. What makes this first task challenging is that the wireless network (WA) system must be safe, secure and as reliable as current wired systems. Once completed, we envisage a new program in the development of pervasive wireless network resources for anesthetists and surgical practice in general. In these cases being hands-free, being able to access information, direct activities by simple movement, gestures adds to more efficient and clean surgical practice. \\

The project involves two parts: 

\begin{enumerate}
	\item Designing and programming hardware nodes (sensors) to collect wireless sensor data, and 
	
	\item Software application development to visualize the data in real-time. The project requires developing a wireles data analysis platform.
\end{enumerate}


\section{Anaesthetic Parameters for Patient Monitoring}

Countless operations are conducted on a daily basis which requires patients to be under general anaesthesia. To ensure the patient's optimal safety when under anaesthesia, it is necessary to consistently monitor certain parameters to ensure that they remain within a specified range which is considered to be safe or normal. Based on Atlee's Complications in Anesthesia \cite{atlee2006complications}, monitoring of anaesthetic admisnistration reduces the probability of anaesthetic overdose or underdose. 

Complications associated with an improper dosage of anaesthesia could arise such as coma, brain damage, nerve damage, or possibly death, if real-time observations of vital signs are not conducted. The risk of such issues occuring during surgery could be reduced significantly if a feedback system were implemented for the amount of anaesthetics administered to the patient. Such parameters have been stipulated by the ANZCA \cite{anaesthesiaguide}. 

\subsection{Theory of Electrocardiography}

The electrocardiogram (ECG) refers to the recording of the "differences in electrical potential generated by the heart" using electrodes which are placed on the surface of the skin \cite{noble1990electrocardiography}. Both the action potentials generated by individual cells and sequence of activation affect the signal registered during electrocardiography \cite{noble1990electrocardiography}. Other factors which alter the final signal include "the position of the heart within the body, the nature of the intervening tissue, and the distance to the recording electrode" \cite{noble1990electrocardiography}. Despite the many factors which can possibly contribute a change to the electrocardiogram, it is still possible to deduce with high accuracy the state of the heart from the surface ECG due to "the careful correlation of electrocardiographic patterns with observed anatomic, pathologic, and physiologic data" \cite{noble1990electrocardiography}. 


\begin{figure}[H]
	\centering
	\includegraphics[width=0.7\linewidth]{ch3f3.jpg}
	\caption{The Cardiac Depolarization Route. AVN: Atrioventricular Node; SAN: Sinoatrial Node. \cite{hall2015guyton}}
	\label{cardiacdepolarization}
\end{figure}

\begin{figure}[H]
	\centering
	\includegraphics[width=0.7\linewidth]{ch3f4.jpg}
	\caption{The Basic Pattern of Electrical Activity across the Heart \cite{ashley2004conquering}}
	\label{ecgpattern}
\end{figure}

Figure \ref{ecgpattern} shows a graphical representation of a typical electrocardiograph trace of the electrical signals from the heart. 

The basic ECG pattern is correlated as follows: 

\begin{itemize}
	\item "Electrical activity towards a lead causes an upward deflection" \cite{ashley2004conquering}
	\item "Electrical activity away from a lead causes a downward deflection" \cite{ashley2004conquering} 
	\item "Depolarization and repolarization deflections occur in opposite directions" \cite{ashley2004conquering} 
\end{itemize}  

The abstract below by Ashley and Niebauer (2004) (p.g. 19) \cite{ashley2004conquering} succinctly explains the types of waves and intervals of the ECG trace, (which mainly comprises of three different waves, namely P, QRS complex, and T): 

\blockquote{
"The {\bf P wave} is a small deflection wave that represents atrial depolarization. 

The {\bf PR interval} is the time between the first deflection of the P wave and the first deflection of the QRS complex. 

The three waves of the {\bf QRS complex} represent ventricular depolarization. For the inexperienced, one of the most confusing aspects of ECG reading is the labeling of these waves. The rule is: if the wave immediately after the P wave is an upward deflection, it is an R wave; if it is a downward deflection, it is a Q wave:

\begin{itemize}
	\item small {\bf Q waves} correspond to depolarization of the interventricular septum. Q waves can also relate to breathing and are generally small and thin. They can also signal an old myocardial infarction (in which case they are big and wide)
	\item the {\bf R wave} reflects depolarization of the main mass of the ventricles – hence it is the largest wave
	\item the {\bf S wave} signifies the final depolarization of the ventricles, at the base of the heart 
\end{itemize}

The {\bf ST segment}, which is also known as the ST interval, is the time between the end of the QRS complex and the start of the T wave. It reflects the period of zero potential between ventricular depolarization and repolarization. 

{\bf T waves} represent ventricular repolarization (atrial repolarization is obscured by the large QRS complex)." }


\subsection{Theory of Electroencephalography}

An electroencephalogram (EEG) is a non-invasive test that detects electrical activity in your brain using small, flat metal discs (electrodes) attached to your scalp. The brain cells communicate using electrical impulses, and such activity will be showing up as wavy lines on an EEG recording. EEG measures voltage fluctuations resulting from ionic current within the neurons of the brain. It is most often used to diagnose epilepsy, which will cause the patient to have an abnormal EEG reading but also used to detect coma, sleep disorders, brain death etc \cite{jiahui1}.  

Generally, EEG is used to evaluate several types of brain disorders. In surgery context, it is commonly used to determine the overall electrical activity of the brain as well as monitoring blood flow. EEG is also used in quantifying and characterizing effects of anaesthetic agents on the central nervous system, providing valuable information to the anaesthesiologist during the surgery \cite{jiahui2}. 

Through an EEG diagram, depth of anaesthesia as well as the effect of different sedative at different period of surgery will be able to be monitored, especially during medically induced coma.  

\begin{figure}[H]
	\centering
	\includegraphics[width=\linewidth]{jiahuipic1.jpg}
	\caption{The Structure of EEG Systems \cite{jiahui2}}
\end{figure}

A raw EEG signal will require digitization and computerization using modern technology to be able to generate specific EEG parameters which can be used to correlate to the effectiveness of anaesthetic agents in alteration of consciousness and inducing coma. Useful information can be extracted in both the time and frequency domain, where the frequency domain approach produces a better precise description. 

During surgical procedure, anaesthetic agents are administered to ensure that patients can tolerate unpleasant and painful interventions to avoid complications. This is achieved through a combination of different effects of several agents, and the alteration of consciousness (hypnotic effect), immobility (through muscle relaxation) and limitations to reactions to surgical stimulation (anti-nociception). EEG recording is primarily used as a source to measure hypnosis and anti-nociception. In short, EEG provides a huge amount of information to the anaesthesiologist in understanding of mechanism of anaesthesia. 

\subsubsection{Engineering Understanding}

A typical adult human EEG signal is about 10uV to 100uV in amplitude when measured from the scalp and is about 1-2mV when measured from the subdural electrodes. 

EEG typically contains different rhythmic activity as well as transients and can be categorize into a few different bands as illustrated below

\begin{enumerate}
	\item Delta wave – at below 4Hz
	\item Theta wave – at 4 - 7Hz
	\item Alpha wave – at 8 – 15 Hz
	\item Beta wave – at 16 – 31 Hz
	\item Gamma wave – at 32++ Hz
	\item Mu wave – at 8 – 12 Hz
\end{enumerate}


Each of the bands correlates to different functionality and regions of brains, and will be used pathologically to monitor different areas of interest. However, it is vital to know that EEG is affected by age, and EEG in childhood has lower frequency oscillations than adult EEG and have to be taken into consideration during analysis \cite{jiahui3}. 

Under the use of anaesthetics, different general anaesthesia will result in different effects on the EEG observed. For example, the use of propofol will result in a rapid alpha and nonreactive EEG pattern seen over most of the scalp \cite{jiahui4}. 

Due to the small amplitude of EEG signal, it is very prone to artefact and noise in some electrical activities arising from other sources other than the brain that are able to modify, distort or cancel the brain activity. This can be categorized into physiological artifacts (e.g. cardiac activity, muscle activity, eye movement) and non-physiological artifacts (e.g. electrode artifacts, external devices etc.) \cite{jiahui5}.


\subsection{Theory of Photoplethysmography (PPG)}

As a crucial part of patient monitoring system, PPG sensor, known as pulse oximeter is used to measure the SpO2 level. “SpO2 stands for peripheral capillary oxygen saturation, an estimate of arterial oxygen saturation, or SaO2, which refers to the amount of oxygenated haemoglobin in the blood" \cite{george1}.

The blood oxygen saturation reading of a normal person should be between 95\% and 100\% \cite{george1}. A respiratory or cardiovascular problem may be present if the oxygen saturation drops to 90\% ~ 95\%. The patient will highly likely experience hypoxic if the oxygen saturation falls under 90\% \cite{george2}.

\subsubsection{Working Principle of the Infrared PPG Sensor}

Photoplethysmogram Sensor (PPG sensor) takes advantage of the different absorption level of Oxyhemoglobin (HbO2) and deoxyhemoglobin (Hb) for the light beams with different wavelength. By placing a pair of infrared and red LED on one side of the finger and a receiver on the other side, the variation of light intensity can be measured. The light transmitting through fingertip will be absorbed by pulsatile arterial blood, non pulsatile arterial blood, venous blood and tissue \cite{george5}. As we know the pulsatile arterial blood is varying according to the heart pulse, hence the light absorption is also varying according to the heart pulse. Such variation is used as the waveform of PPG signal.

\begin{figure}[H]
	\centering
	\includegraphics[width=0.5\linewidth]{georgepic1.jpg}
	\caption{Waveform of PPG Signal \cite{george6}}
\end{figure}

\begin{figure}[H]
	\centering
	\includegraphics[width=0.5\linewidth]{georgepic2.jpg}
	\caption{Absorption Spectra of Hemoglobin \cite{george3}}
\end{figure}

The Oxygen saturation level can be calculated by the following equation:
\begin{equation}
	SpO_2=\frac{HbO_2}{HbO_2 + Hb}
\end{equation}

Where: 
\begin{itemize}
	\item $HbO_2$: Oxyhemoglobin
	\item $Hb$: Deoxyhemoglobin
\end{itemize}

The pulse oximeter uses Beer–Lambert law \cite{george4} \cite{george5} to define relationship between the light attenuation (light absorption) and the material it travels through \cite{george7}.
  
A normalized ratio of the Red to IR then can be calculated:

\begin{equation}
	R = \frac{\frac{AC_{RED}}{DC_{RED}}}{\frac{AC_{IR}}{DC_{IR}}}
\end{equation}

Where: 
\begin{itemize}
	\item $AC_{RED}$: AC Component of Red light waveform
	\item $DC_{RED}$: DC Component of Red light waveform
	\item $AC_{IR}$: AC Component of Infrared light waveform
	\item $DC_{IR}$: DC Component of Infrared light waveform
\end{itemize}

The accurate SpO2 is proportional to this ratio. Proper calibration can be done to get the accurate SpO2 reading \cite{george10}. The PPG waveform is periodic and its period is based on the real time heart rate. Therefore, the heart rate can also be measured through signal processing of the PPG waveform. 

\subsection{Theory of Blood Temperature}

The core temperature of the body is regulated to be at a constant value of $36.9 \degree C$. When anaesthetics are administered, the temperature of the body drops due to "an internal core-to-peripheral redistribution of body heat" \cite{sessler2008temperature}. This has to be monitored carefully to ensure that there is no overdose of anaesthetics administered. The onset of such an issue can be avoided through the "detection of malignant hyperthermia" and "quantification of hyperthermia and hypothermia" \cite{sessler2008temperature}. 

The monitoring of blood temperature in the case of this project is done through direct contact with the skin. Although the actual blood temperature is not measured (through invasive means), the body temperature gives a rough indication of what the blood temperature is. This is a reasonable assumption made for this project as a proof of concept. 
