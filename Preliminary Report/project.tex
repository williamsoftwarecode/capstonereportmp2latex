\chapter{Project Design and Development}


\section{System Overview}

The hardware of this vital signs monitoring system has been divided into three major sections as seen in Figure 3.1, namely: 

\begin{itemize}
	\item Sensors - Collects raw measurements of separate vital sign parameters 
	\item Terminals - Processes data and handle transmission/reception of information
	\item Display - Visualises output of critical vital sign information
\end{itemize}

A block diagram of the Wireless Vital Signs Monitoring System is shown in Figure \ref{system} down below. The system consists of three major portions: the Sensors which collect the raw measurements of separate vital sign parameters, the Terminals which process the data and handle transmission/reception of information, and the Display which visualises the output. The details of the development and options considered for each portion are further described in Sections \ref{terminals}, \ref{sensors}, and \ref{processing}.

\begin{figure}[H]
	\centering
	\includegraphics[width=\linewidth]{system2.png}
	\caption{Wireless Vital Signs Monitoring System Block Diagram}
	\label{system}
\end{figure}

\section{Design Specifications}
\label{specifications}

\section{Hardware Design and Development}







\section{Software Design and Development}
\label{software}




\section{Analysis of Competitors - Similar Papers - Literature Review}







\section{Analysis of Existing Systems}

\subsection{Consideration of Bionomadix}

\subsection{Consideration of Libelium and Waspmote}

\subsection{Consideration of Cooking Hacks}
