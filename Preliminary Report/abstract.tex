
\chapter*{Abstract}

This report documents the design and development process of a wireless vital signs monitoring system by Master of Engineering (Electrical) students at the University of Melbourne. The project aims to assemble a completely wireless system with attached sensors which is as reliable and accurate as current conventional monitoring systems.  

In modern surgery, there is an ever increasing number of devices attached to a patient in the surgery room. The presence of wire bundles introduces implicit hazards which could possibly harm the patient or the surgeons. To mitigate and circumvent this issue, the number of wires could be reduced through the implementation of wireless vital signs monitoring systems in operating theatres.  

The wireless vital signs monitoring system consists of a main operating terminal and connected sensors. The anaesthetic parameters chosen for observation are electrocardiography (ECG), electroencephalography (EEG), photoplethysmography (PPG), blood temperature, and blood pressure. 

The terminal is designed for data processing, data transmission, and data visualisation. To do this, the Raspberry Pi 3 Model B was chosen due to its desirable properties as a miniaturised computer and fulfills the platform requirements of the project. It runs applications on an ARM processor loaded with a Linux-based Raspbian Jessie operating system. In terms of wireless transmission, it is fully capable of WiFi transmission and establishing a VNC server for secured external access. 

The sensors are designed individually with analog inputs and these signals are fed into analog-to-digital converters. The converted digital signals are then processed by applications on the terminal to be visualised and then transmitted wirelessly to another terminal for displaying the output. 

Both the terminal and sensors have been tested. The terminal has successfully collected signals from different sensors and transmitted them across to a second platform to be displayed. Additionally, the sensors circuit testing shows multiple sensors to be fully working and accurate. 

