
\chapter*{Abstract}

This  thesis considers the  problem of the rejection of
exogenous  disturbance   signals  applied to a linear time  invariant  plant.
An   analysis of the disturbance  rejection  performance of
periodically  time varying feedback controllers is  conducted. The performance  used
is the  induced system norm of the  closed loop  disturbance response  system.


The analysis  considers   both  discrete  systems, in  which both
plant and controller  operate in  discrete-time,   and also
sampled-data  systems,  in which the  plant operates in
continuous-time, while the  controller  operates  in
discrete-time.  The disturbance  input  signals  under consideration are  assumed to
belong to the  $l_p$ signal spaces (for discrete  systems) and   the  $L_p$ signal spaces
(for sampled-data  systems).

In the  analysis, time  invariant  controllers are distinguished from strictly
periodically  time varying controllers. This allows a  comparison to be made   of the
relative  disturbance rejection  performance of these two  classes  of controllers.
 For a given nonlinear  time invariant periodic controller, a nonlinear time invariant
controller is constructed which  stabilizes the closed loop system
when applied to the linear time  invariant plant. Necessary and
sufficient conditions are presented under which the nonlinear
time invariant controller gives strictly better  disturbance rejection performance
than the nonlinear  time invariant periodic controller.


Earlier results on $l_2$ (and  $L_2$) performance  of linear periodic  controllers
are extended  to  present a unified treatment of $l_p$ performance for all $p \in
[1,\infty]$ (respectively, $L_p$ performance for all $p \in[1,\infty]$).
  Results  are  obtained  for both linear and nonlinear controllers.  Thus results in
the  literature  on the inferior disturbance rejection performance  of linear
periodic  controllers are shown to remain valid   when the class of  controllers  is
extended to include nonlinear controllers.  Thus the principal claims  to originality
of this thesis are that  it obtains results  for   a wider  class  of disturbance
signals, and  that it provides a performance comparison of nonlinear time varying
controllers with nonlinear time invariant controllers.

