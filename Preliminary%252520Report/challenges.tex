\chapter{Discussion}

\section{Challenges and Limitations}

\section{Power Efficiency}

According to Sohraby, Minoli, and Taieb \cite{sohraby2007wireless}, power efficiency can be achieved by having: 

\begin{enumerate}
	\item "Low-duty-cycle operation."
	\item "Local/in-network processing to reduce data volume (and hence transmission time)."
	\item "Multihop networking reduces the requirement for long-range transmission since signal path loss is an inverse exponent with range or distance. Each node in the sensor network can act as a repeater, thereby reducing the link range coverage required and, in turn, the transmission power." 
\end{enumerate}

\section{Interference}

One of the major problems identified with wireless systems in general is the high probability of interference. 

\section{Security Risk}

\section{Data Management}

\section{Regulation and Compliance Standards}

\section{Reliability}
Raspberry Pi has "uptime"

\section{Range}


\section{Raspberry Pi USB}

Only a single root USB port is available on the Raspberry Pi 3 Model B and all data traffic from USB devices are directed to this single bus \cite{rpi3hardware}. The maximum speed of this root USB port is 480Mbps \cite{rpi3faqs}. The WVSMT utilises all 4 USB ports available for connections to individual Sensors and this could possibly lead to bottlenecking in the root USB port. However, it is noted that the bandwidth required for each sensor is low and that this scenario is not probable. 

\section{Intel Nuc}